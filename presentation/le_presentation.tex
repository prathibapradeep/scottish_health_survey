% Options for packages loaded elsewhere
\PassOptionsToPackage{unicode}{hyperref}
\PassOptionsToPackage{hyphens}{url}
%
\documentclass[
]{article}
\usepackage{amsmath,amssymb}
\usepackage{lmodern}
\usepackage{ifxetex,ifluatex}
\ifnum 0\ifxetex 1\fi\ifluatex 1\fi=0 % if pdftex
  \usepackage[T1]{fontenc}
  \usepackage[utf8]{inputenc}
  \usepackage{textcomp} % provide euro and other symbols
\else % if luatex or xetex
  \usepackage{unicode-math}
  \defaultfontfeatures{Scale=MatchLowercase}
  \defaultfontfeatures[\rmfamily]{Ligatures=TeX,Scale=1}
\fi
% Use upquote if available, for straight quotes in verbatim environments
\IfFileExists{upquote.sty}{\usepackage{upquote}}{}
\IfFileExists{microtype.sty}{% use microtype if available
  \usepackage[]{microtype}
  \UseMicrotypeSet[protrusion]{basicmath} % disable protrusion for tt fonts
}{}
\makeatletter
\@ifundefined{KOMAClassName}{% if non-KOMA class
  \IfFileExists{parskip.sty}{%
    \usepackage{parskip}
  }{% else
    \setlength{\parindent}{0pt}
    \setlength{\parskip}{6pt plus 2pt minus 1pt}}
}{% if KOMA class
  \KOMAoptions{parskip=half}}
\makeatother
\usepackage{xcolor}
\IfFileExists{xurl.sty}{\usepackage{xurl}}{} % add URL line breaks if available
\IfFileExists{bookmark.sty}{\usepackage{bookmark}}{\usepackage{hyperref}}
\hypersetup{
  pdftitle={R Notebook},
  hidelinks,
  pdfcreator={LaTeX via pandoc}}
\urlstyle{same} % disable monospaced font for URLs
\usepackage[margin=1in]{geometry}
\usepackage{graphicx}
\makeatletter
\def\maxwidth{\ifdim\Gin@nat@width>\linewidth\linewidth\else\Gin@nat@width\fi}
\def\maxheight{\ifdim\Gin@nat@height>\textheight\textheight\else\Gin@nat@height\fi}
\makeatother
% Scale images if necessary, so that they will not overflow the page
% margins by default, and it is still possible to overwrite the defaults
% using explicit options in \includegraphics[width, height, ...]{}
\setkeys{Gin}{width=\maxwidth,height=\maxheight,keepaspectratio}
% Set default figure placement to htbp
\makeatletter
\def\fps@figure{htbp}
\makeatother
\setlength{\emergencystretch}{3em} % prevent overfull lines
\providecommand{\tightlist}{%
  \setlength{\itemsep}{0pt}\setlength{\parskip}{0pt}}
\setcounter{secnumdepth}{-\maxdimen} % remove section numbering
\ifluatex
  \usepackage{selnolig}  % disable illegal ligatures
\fi

\title{R Notebook}
\author{}
\date{\vspace{-2.5em}}

\begin{document}
\maketitle

\hypertarget{life-expectancy}{%
\section{Life Expectancy}\label{life-expectancy}}

\hypertarget{the-data}{%
\section{The Data}\label{the-data}}

\begin{itemize}
\tightlist
\item
  What is life expectancy and why it was chosen? Life expectancy is a
  statistical measure of the average time an organism is expected to
  live, based on the year of its birth, its current age and any other
  demographic factors including sex and geographic area. It is a useful
  statistic as it provides a snapshot of the health of a population and
  allows identification of inequalities between populations, or
  geographic areas within a country. Life expectancy is also used to
  inform pensions policy, research and teaching.
\end{itemize}

It does not, however, reflect how long a person will survive as it does
not take into account changes in health care or other social factors
which may occur during someone's lifetime.

\begin{itemize}
\item
  The source of the data The dataset used was published by the Scottish
  Government (statistics.gov.uk) and contains information on life
  expectancy from birth and in age groups of 4 years until 90 years,
  geographic area data, sex, for a period from 1991 through to 2019.
\item
  Why was this data selected? The data fits the brief and contain
  sufficient information in order to arrive at, hopefully, meaningful
  conclusions and observations. The data is designated as National
  Statistics by the UK Statistics Authority ensuring that the data meet
  the highest standards of trustworthiness, quality and public value.
\end{itemize}

\hypertarget{the-trend}{%
\section{The Trend}\label{the-trend}}

Unsurprisingly life expectancy, for a developed country such as
Scotland, has increased over time.

We also have available the 95\% confidence limits which is shown as a
grey ribbon superimposed over each trend. Which is the the range of
values that the actual life expectancy value of the population is likely
to lie within. Estimates from larger populations, such as Scotland as a
whole, has smaller confidence intervals and are therefore more accurate
estimates than compared with smaller populations such as local
authorities.

Dumfries \& Galloway - my home area! - Male life expectancy has
increased from approx. 72.5 years in 1991-1993 to 78 years in 2017-2019
An increase of 5.5 yrs. - For females, that increase has been from
approx. 77 years in 1991-1993 to just over 80 years in the period
2017-2019 An increase of 3 yrs.

Scotland - Life expectancy for males has increased from 71.5 years in
1991-1993 to 77.5 years in 2017-2019 An increase of 6 years - For
females, life expectancy has increased from 76.5 years to 81.5 years for
the same time period. An increase of 5 years

For Scotland as a whole, we see that life expectancy has remained
virtually unchanged since 2013-2015. This is evident in most of
Scotland's council areas where life expectancy has stalled since
2013-2015 with some areas showing a decreasing life expectancy.

The reasons for this are complicated.

Why this is the case is a complicated subject with many moving parts -
but it may be attributed to cuts in social care, NHS spending, removal
of subsidies, the global financial crisis in 2008, and drug and alcohol
abuse.

\hypertarget{rank}{%
\section{Rank}\label{rank}}

highest life expectancy by local authority and health board lowest life
expectancy by local authority and health board

\end{document}
